% !TEX encoding   = UTF8
% !TEX root       = manual.tex
% !TEX spellcheck = en_GB

\chapter{Customizing \Tw}

\section{Syntax highlighting}

Among its many other features, {\Tw} also include syntax highlighting\index{syntax highlighting}. This means that certain things like {\LaTeX} commands, environments, or comments are coloured, underlined, or highlighted in some other way. {\Tw} also provides the ability to switch between different highlighting schemes\footnote{Use \menu{Format}\submenu\menu{Syntax Coloring} to change the highlighting scheme for the current document, and \menu{Edit}\submenu{Preferences\dots}\submenu\menu{Editor}\submenu\menu{Syntax Coloring} to set the default one.}, and to define your own ones. This is useful if you often work with types of files for which no highlighting scheme is provided by default, or if you want to adjust the highlighting schemes to better match your system's colour scheme.

To modify the highlighting schemes, you have to edit the plain-text file \path{<resources>/configuration/syntax-patterns.txt}. This file can contain any number of individual sections, each defining a single highlighting scheme to be displayed in the menu structure of {\Tw}. To define a section, just write the name enclosed in square brackets on a line of its own. Naturally, these names should not include the \verb|]| character. By default, the following two sections are defined:
\begin{verbExample}
[LaTeX]
[ConTeXt]
\end{verbExample}
In addition, you can add comments to the file by starting a line with \verb|#|. Empty lines are ignored.

Each section consists of an arbitrary number of styling rules. Each such instruction consists of three parts: a formatting instruction, a spell-check flag, and a regular expression\footnote{For some details on regular expressions, see \ref{sec:regexp}} defining what part of a text to match. These parts must all be on the same line, and separated by whitespaces (e.g., spaces or tabstop characters). Take for example the following line from the default \verb|LaTeX| section:
\begin{verbExample}
red    Y    %.*
\end{verbExample}
The first part, \verb|red|, defines the format (in this case, a red foreground colour is specified). The second part, \verb|Y|, defines that spellchecking should be enabled for text that matches this particular rule. Sometimes, it is useful to put \verb|N| here to disable spellchecking. For example, if spellchecking would be enabled for {\LaTeX} commands, most documents would be flooded with red underlines indicating misspelled words when in fact they are only special commands. Finally, the third part specifies that this rule should be applied to all text preceded by \verb|%|.

Let us take a closer look at the three parts of each rule. In its most general form, the first part---the format instruction---looks like 
\begin{verbExample}
<foreground_colour>/<background_colour>;<fontflags>
\end{verbExample}
The \verb|<fontflags>| can be specified independent of the colours (note, though, that it must always be preceded by a \verb|;|). The background colour (together with the \verb|/|) can be omitted, but if you specify it, you also have to specify the foreground colour.

Each colour can either be specified by an SVG name\footnote{See \url{http://www.w3.org/TR/SVG/types.html#ColorKeywords} for a list of valid names.} or by a hexadecimal value (\verb|#rrggbb|) similar as in web documents. The \verb|<fontflags>| can be any combination of the letter \verb|B| (bold), \verb|I| (italic), and \verb|U| (underlined).

Examples of valid formatting instructions are:
\begin{verbExample}
red
white/#000000
;B
blue;I
#000000/#ffff00;U
\end{verbExample}

\section{Keyboard shortcuts}
\label{sec.shortcuts}

The use of keyboard shortcuts\index{keyboard shortcuts} greatly facilitates typing in and the management of the source and the preview windows. Their use is much more effective than the use of buttons for frequently-used actions.

Below, you'll find the shortcuts for source and preview windows. Note that on Mac OS X, \verb|Ctrl| actually refers to the \emph{Command key}, which is the usual modifier for keyboard shortcuts. Although the keyboard shortcuts are specified with \verb|Ctrl|, this will appear as the \emph{Command-key} symbol in menus. (To refer to the actual \emph{Control key} on the Mac, the shortcut file should use the name \verb|Meta|).

Note that the shortcuts listed below are the default shortcuts for the English interface of {\Tw}. Different languages may use different shortcuts.

All the shortcuts can be redefined either to create new shortcuts or to modify the existing ones to match personal uses or change shortcuts not adapted to one particular keyboard layout. The list of possible actions\index{keyboard shortcuts!actions} to associate with shortcuts is given after the predefined shortcuts.

To define your own shortcuts, put a file named \path{shortcuts.ini}\index{keyboard shortcuts!shortcuts.ini} in the \path{<resources>/configuration} folder, next to \path{auto-indent-patterns.txt}, \path{delimiter-pairs.txt}, \dots, \path{texworks-config.txt}.

For example, this  file could contain:
\begin{verbExample}
actionHard_Wrap = Shift+F3
actionLast_Page = Ctrl+End
actionFirst_Page = Ctrl+Home
actionWrap_Lines = F3
actionLine_Numbers = F4
actionBalance_Delimiters = F9
\end{verbExample}

The first line defines that using \keysequence{Shift+F3} should open the hardwrap dialogue box in the source window; the second (\keysequence{Ctrl+End}) should bring you to the last page and \keysequence{Ctrl+Home} (third line) should take you to the first page; with \keysequence{F3} you want to wrap/unwrap lines in the source, with \keysequence{F4} you will show/hide line numbers and with \keysequence{F9} you intend to select the text between corresponding delimiters in the source.

\needspace{7\baselineskip}
\subsection{Predefined shortcuts}\index{keyboard shortcuts!predefined}\index{shortcuts|see {keyboard shortcuts}}

For working in the source window:
\input shortcutsTeXDocument.tex

\needspace{5\baselineskip}
Moving the cursor (hold \keysequence{Shift} to select):
\begin{longtable}{Pl}
\toprule
Shortcut & Action \\
\midrule \endhead
$\rightarrow$      & 1 character right \\
Ctrl+$\rightarrow$ & 1 word right \\
$\leftarrow$       & 1 character left \\
Ctrl+$\leftarrow$  & 1 word left \\
$\uparrow$         & 1 line up \\
$\downarrow$       & 1 line down \\
PgUp               & 1 screen up \\
PgDown             & 1 screen down \\
Home               & Begin of line \\
Ctrl+Home          & Begin of document \\
End                & End of line \\
Ctrl+End           & End of document \\
\bottomrule
\end{longtable}

For working in the preview window:
\input shortcutsPDFDocument.tex

\subsection{Actions listed alphabetically}
\index{actions!alphabetically}
\begin{longtable}{QQ}
\toprule
actionAbout\_Scripts       & actionPaste \\
actionAbout\_TW            & actionPlace\_on\_Left \\
actionActual\_Size         & actionPlace\_on\_Right \\
actionApply\_to\_Selection & actionPreferences \\
actionAutoIndent\_None     & actionPrevious\_Page \\
actionAuto\_Follow\_Focus  & actionQuit\_TeXworks \\
actionBalance\_Delimiters  & actionRedo \\
actionClear                & actionRemove\_Aux\_Files \\
actionClose                & actionReplace \\
actionComment              & actionReplace\_Again \\
actionCopy                 & actionRevert\_to\_Saved \\
actionCopy\_to\_Find       & actionSave \\
actionCopy\_to\_Replace    & actionSave\_All \\
actionCut                  & actionSave\_As \\
actionFind                 & actionScroll \\
actionFind\_Again          & actionSelect\_All \\
actionFind\_Selection      & actionSelect\_Image \\
actionFirst\_Page          & actionSelect\_Text \\
actionFit\_to\_Width       & actionShow\_Hide\_Console \\
actionFit\_to\_Window      & actionShow\_Scripts\_Folder \\
actionFont                 & actionShow\_Selection \\
actionFull\_Screen         & actionSide\_by\_Side \\
actionGoToHomePage         & actionSmartQuotes\_None \\
actionGo\_to\_Line         & actionStack \\
actionGo\_to\_Page         & actionSyntaxColoring\_None \\
actionGo\_to\_Preview      & actionTile \\
actionGo\_to\_Source       & actionTo\_Lowercase \\
actionHard\_Wrap           & actionTo\_Uppercase \\
actionIndent               & actionToggle\_Case \\
actionLast\_Page           & actionTypeset \\
actionLine\_Numbers        & actionUncomment \\
actionMagnify              & actionUndo \\
actionManage\_Scripts      & actionUnindent \\
actionNew                  & actionUpdate\_Scripts \\
actionNew\_from\_Template  & actionWrap\_Lines \\
actionNext\_Page           & actionWriteToMailingList \\
actionNone                 & actionZoom\_In \\
actionOpen                 & actionZoom\_Out \\
actionOpen\_Recent         & \\
\bottomrule
\end{longtable}


\subsection{Actions listed by menu}
\index{actions!by menu}

For the source window:
\input menuactionsTeXDocument.tex

\bigskip
For the preview window:
\input menuactionsPDFDocument.tex

\subsection{Other actions}
In addition to the static actions listed above, there are also actions for scripts. These are dynamic in nature, as they are created on-the-fly for the available scripts (which may change when you install scripts, remove them, or change some settings). All of these actions are of the form \verb*|Script: <script_title>|, where \verb*|<script_title>| must be replaced appropriately. If you have a script that shows up as \verb|My Script|, for example, the corresponding action would be named \verb*|Script: My Script|.


\section{Roots for completion}
\index{auto-completion!roots}\index{completion|see {auto-completion}}
\label{sec.autocompletion}

We give here the keywords for auto-completion as they are supplied by {\Tw}. They are given in the files \path{tw-basic.txt}, \path{tw-context.txt} (initially empty) and \path{tw-latex.txt} in the \path{<resource>\completion} folder.

We give them in three columns: the first two show the keywords, the third the {\AllTeX} code produced. In some cases there is only the code, this means that you can start to enter the {\AllTeX} code and try to complete it with \keysequence{Tab}.

\newcommand{\AutoCompRet}{$\mathcal{R}$}
\newcommand{\AutoCompIns}{$\mathcal{I}$}
During completion, the system inserts line feeds and puts the cursor at the first place where one has to enter information to complete the typing. To represent the line feeds we used {\AutoCompRet} and {\AutoCompIns} for the input point. 

\needspace{4\baselineskip}
So, a line like ``\textbackslash begin\{abstract\}{\AutoCompRet}{\AutoCompIns}{\AutoCompRet}\textbackslash end\{abstract\}•'' should be interpreted as
\begin{verbExample}
\begin{abstract}

\end{abstract}•
\end{verbExample}
with the cursor being position on the central, empty line.

It is possible to see that the keywords have some pattern. The mathematical variables have a keyword starting with \verb|x|, when they are in a mathematical environment; when they are used alone in the text you add \verb|d| in front. For example, \verb|xa| and \verb|dxa| give \verb|\alpha|, if there is a capital there is a \verb|c|, as  \verb|xo| for \verb|\omega| and \verb|xco| for \verb|\Omega|. The keywords for environments start with \verb|b|: \verb|bali| for \verb|\begin{align}| (\verb|b| is a mnemonic for \verb|\begin|). When the environment has possible options, there is one or more \verb|o| added to the base name: \verb|bminp| gives \verb|\begin{minipage}{}...| while \verb|bminpo| gives \verb|\begin{minipage}[]{}...|.

%\vspace{12pt}
%
Keywords defined in \path{tw-basic.txt}\index{auto-completion!tw-basic.txt} (defined in \TeX):
%\input{tw-basic_fr.tex}
\begin{longtable}{>{\footnotesize}p{15mm}>{\footnotesize}p{15mm}>{\footnotesize}p{95mm}}
\toprule
xa   & \textbackslash xa   & \textbackslash alpha \\
xb   & \textbackslash xb   & \textbackslash beta \\
     & \textbackslash bsk  & \textbackslash bigskip \\
     &                     & \textbackslash bigskip{\AutoCompRet} \\
xch  & \textbackslash xch  & \textbackslash chi \\
xd   & \textbackslash xd   & \textbackslash delta \\
xcd  & \textbackslash xcd  & \textbackslash Delta \\
xe   & \textbackslash xe   & \textbackslash epsilon \\
xet  & \textbackslash xet  & \textbackslash eta \\
xg   & \textbackslash xg   & \textbackslash gamma \\
xcg  & \textbackslash xcg  & \textbackslash Gamma \\
     &                     & \textbackslash hskip \\
     &                     & \textbackslash indent \\
     &                     & \textbackslash input \\
xio  & \textbackslash xio  & \textbackslash iota \\
xl   & \textbackslash xl   & \textbackslash lambda \\
xcl  & \textbackslash xcl  & \textbackslash Lambda \\
     & \textbackslash msk  & \textbackslash medskip \\
     &                     & \textbackslash medskip{\AutoCompRet} \\
xm   & \textbackslash xm   & \textbackslash mu \\
     &                     & \textbackslash noindent \\
xn   & \textbackslash xn   & \textbackslash nu \\
xo   & \textbackslash xo   & \textbackslash omega \\
xco  & \textbackslash xco  & \textbackslash Omega \\
     &                     & \textbackslash par \\
xcph & \textbackslash xcph & \textbackslash Phi \\
xph  & \textbackslash xph  & \textbackslash phi \\
xp   & \textbackslash xp   & \textbackslash pi \\
xcp  & \textbackslash xcp  & \textbackslash Pi \\
xcps & \textbackslash xcps & \textbackslash Psi \\
xps  & \textbackslash xps  & \textbackslash psi \\
xr   & \textbackslash xr   & \textbackslash rho \\
     &                     & \textbackslash scriptsize \\
xs   & \textbackslash xs   & \textbackslash sigma \\
xcs  & \textbackslash xcs  & \textbackslash Sigma \\
     &                     & \textbackslash smallskip{\AutoCompRet} \\
     & \textbackslash ssk  & \textbackslash smallskip{\AutoCompRet} \\
xt   & \textbackslash xt   & \textbackslash tau \\
tex  & \textbackslash tex  & \textbackslash TeX \\
     &                     & \textbackslash TeX \\
texs & \textbackslash texs & \textbackslash TeX\textbackslash  \\
     &                     & \textbackslash TeX\textbackslash  \\
xth  & \textbackslash xth  & \textbackslash theta \\
xcth & \textbackslash xcth & \textbackslash Theta \\
xu   & \textbackslash xu   & \textbackslash upsilon \\
xcu  & \textbackslash xcu  & \textbackslash Upsilon \\
xve  & \textbackslash xve  & \textbackslash varepsilon \\
xvph & \textbackslash xvph & \textbackslash varphi \\
xvp  & \textbackslash xvp  & \textbackslash varpi \\
xvr  & \textbackslash xvr  & \textbackslash varrho \\
xvs  & \textbackslash xvs  & \textbackslash varsigma \\
xvth & \textbackslash xvth & \textbackslash vartheta \\
     &                     & \textbackslash vskip \\
xcx  & \textbackslash xcx  & \textbackslash Xi \\
xx   & \textbackslash xx   & \textbackslash xi \\
xz   & \textbackslash xz   & \textbackslash zeta \\
\bottomrule
\end{longtable}


\needspace{6\baselineskip}
Keywords defined in \path{tw-latex.txt}\index{auto-completion!tw-latex.txt} (defined in \LaTeX):
%\input{tw-latex_fr.tex}
\begin{longtable}{>{\footnotesize}p{15mm}>{\footnotesize}p{15mm}>{\footnotesize}p{95mm}}
\toprule
ncol            & \textbackslash ncol      &  \& \\
dd              & \textbackslash dd        & \textbackslash ( {\AutoCompIns} \textbackslash )• \\
dxa             & \textbackslash dxa       & \textbackslash (\textbackslash alpha\textbackslash ) \\
dxb             & \textbackslash dxb       & \textbackslash (\textbackslash beta\textbackslash ) \\
dxch            & \textbackslash dxch      & \textbackslash (\textbackslash chi\textbackslash ) \\
dxd             & \textbackslash dxd       & \textbackslash (\textbackslash delta\textbackslash ) \\
dxcd            & \textbackslash dxcd      & \textbackslash (\textbackslash Delta\textbackslash ) \\
dxe             & \textbackslash dxe       & \textbackslash (\textbackslash epsilon\textbackslash ) \\
dxet            & \textbackslash dxet      & \textbackslash (\textbackslash eta\textbackslash ) \\
dxg             & \textbackslash dxg       & \textbackslash (\textbackslash gamma\textbackslash ) \\
dxcg            & \textbackslash dxcg      & \textbackslash (\textbackslash Gamma\textbackslash ) \\
dxio            & \textbackslash dxio      & \textbackslash (\textbackslash iota\textbackslash ) \\
dxl             & \textbackslash dxl       & \textbackslash (\textbackslash lambda\textbackslash ) \\
dxcl            & \textbackslash dxcl      & \textbackslash (\textbackslash Lambda\textbackslash ) \\
dxm             & \textbackslash dxm       & \textbackslash (\textbackslash mu\textbackslash ) \\
dxn             & \textbackslash dxn       & \textbackslash (\textbackslash nu\textbackslash ) \\
dxo             & \textbackslash dxo       & \textbackslash (\textbackslash omega\textbackslash ) \\
dxco            & \textbackslash dxco      & \textbackslash (\textbackslash Omega\textbackslash ) \\
dxcph           & \textbackslash dxcph     & \textbackslash (\textbackslash Phi\textbackslash ) \\
dxph            & \textbackslash dxph      & \textbackslash (\textbackslash phi\textbackslash ) \\
dxp             & \textbackslash dxp       & \textbackslash (\textbackslash pi\textbackslash ) \\
dxcp            & \textbackslash dxcp      & \textbackslash (\textbackslash Pi\textbackslash ) \\
dxcps           & \textbackslash dxcps     & \textbackslash (\textbackslash Psi\textbackslash ) \\
dxps            & \textbackslash dxps      & \textbackslash (\textbackslash psi\textbackslash ) \\
dxr             & \textbackslash dxr       & \textbackslash (\textbackslash rho\textbackslash ) \\
dxs             & \textbackslash dxs       & \textbackslash (\textbackslash sigma\textbackslash ) \\
                & \textbackslash dxcs      & \textbackslash (\textbackslash Sigma\textbackslash ) \\
dxt             & \textbackslash dxt       & \textbackslash (\textbackslash tau\textbackslash ) \\
dxth            & \textbackslash dxth      & \textbackslash (\textbackslash theta\textbackslash ) \\
dxcth           & \textbackslash dxcth     & \textbackslash (\textbackslash Theta\textbackslash ) \\
dxu             & \textbackslash dxu       & \textbackslash (\textbackslash upsilon\textbackslash ) \\
dxcu            & \textbackslash dxcu      & \textbackslash (\textbackslash Upsilon\textbackslash ) \\
dxve            & \textbackslash dxve      & \textbackslash (\textbackslash varepsilon\textbackslash ) \\
dxvph           & \textbackslash dxvph     & \textbackslash (\textbackslash varphi\textbackslash ) \\
dxvp            & \textbackslash dxvp      & \textbackslash (\textbackslash varpi\textbackslash ) \\
dxvr            & \textbackslash dxvr      & \textbackslash (\textbackslash varrho\textbackslash ) \\
dxvs            & \textbackslash dxvs      & \textbackslash (\textbackslash varsigma\textbackslash ) \\
dxvth           & \textbackslash dxvth     & \textbackslash (\textbackslash vartheta\textbackslash ) \\
dxx             & \textbackslash dxx       & \textbackslash (\textbackslash xi\textbackslash ) \\
dxcx            & \textbackslash dxcx      & \textbackslash (\textbackslash Xi\textbackslash ) \\
dxz             & \textbackslash dxz       & \textbackslash (\textbackslash zeta\textbackslash ) \\
                &                          & \textbackslash addtocounter\{{\AutoCompIns}\}\{•\} \\
                & \textbackslash adc       & \textbackslash addtocounter\{{\AutoCompIns}\}\{•\} \\
adcount         &                          & \textbackslash addtocounter\{{\AutoCompIns}\}\{•\}{\AutoCompRet} \\
                & \textbackslash adl       & \textbackslash addtolength\{{\AutoCompIns}\}\{•\} \\
                &                          & \textbackslash addtolength\{{\AutoCompIns}\}\{•\} \\
adlen           &                          & \textbackslash addtolength\{{\AutoCompIns}\}\{•\}{\AutoCompRet} \\
                &                          & \textbackslash author\{{\AutoCompIns}\}{\AutoCompRet} \\
                &                          & \textbackslash begin\{ \\
babs            & \textbackslash babs      & \textbackslash begin\{abstract\}{\AutoCompRet}{\AutoCompIns}{\AutoCompRet}\textbackslash end\{abstract\}• \\
balis           & \textbackslash balis     & \textbackslash begin\{align*\}{\AutoCompRet}{\AutoCompIns}{\AutoCompRet}\textbackslash end\{align*\}• \\
baliats         & \textbackslash baliats   & \textbackslash begin\{alignat*\}\{{\AutoCompIns}\}{\AutoCompRet}•{\AutoCompRet}\textbackslash end\{alignat*\}• \\
baliat          & \textbackslash baliat    & \textbackslash begin\{alignat\}\{{\AutoCompIns}\}{\AutoCompRet}•{\AutoCompRet}\textbackslash end\{alignat\}• \\
baliedat        & \textbackslash baliedat  & \textbackslash begin\{alignedat\}{\AutoCompRet}{\AutoCompIns}{\AutoCompRet}\textbackslash end\{alignedat\}• \\
baliedato       & \textbackslash baliedato & \textbackslash begin\{alignedat\}[{\AutoCompIns}]{\AutoCompRet}•{\AutoCompRet}\textbackslash end\{alignedat\}• \\
balied          & \textbackslash balied    & \textbackslash begin\{aligned\}\{{\AutoCompIns}\}{\AutoCompRet}•{\AutoCompRet}\textbackslash end\{aligned\}• \\
bali            & \textbackslash bali      & \textbackslash begin\{align\}{\AutoCompRet}{\AutoCompIns}{\AutoCompRet}\textbackslash end\{align\}• \\
bapp            & \textbackslash bapp      & \textbackslash begin\{appendix\}{\AutoCompRet}{\AutoCompIns}{\AutoCompRet}\textbackslash end\{appendix\}• \\
barr            &                          & \textbackslash begin\{array\}{\AutoCompRet}{\AutoCompIns}{\AutoCompRet}\textbackslash end\{array\}• \\
bbmat           & \textbackslash bbmat     & \textbackslash begin\{bmatrix\}{\AutoCompRet}{\AutoCompIns}{\AutoCompRet}\textbackslash end\{bmatrix\}• \\
bcase           & \textbackslash bcase     & \textbackslash begin\{cases\}{\AutoCompRet}{\AutoCompIns}{\AutoCompRet}\textbackslash end\{cases\}• \\
bcent           & \textbackslash bcent     & \textbackslash begin\{center\}{\AutoCompRet}{\AutoCompIns}{\AutoCompRet}\textbackslash end\{center\}• \\
bcenum          & \textbackslash bcenum    & \textbackslash begin\{compactenum\}{\AutoCompRet}\textbackslash item{\AutoCompRet}{\AutoCompIns}{\AutoCompRet}\textbackslash end\{compactenum\}• \\
bcenumo         & \textbackslash bcenumo   & \textbackslash begin\{compactenum\}[{\AutoCompIns}]{\AutoCompRet}\textbackslash item{\AutoCompRet}•{\AutoCompRet}\textbackslash end\{compactenum\}• \\
bcitem          & \textbackslash bcitem    & \textbackslash begin\{compactitem\}{\AutoCompRet}\textbackslash item{\AutoCompRet}{\AutoCompIns}{\AutoCompRet}\textbackslash end\{compactitem\}• \\
bcitemo         & \textbackslash bcitemo   & \textbackslash begin\{compactitem\}[{\AutoCompIns}]{\AutoCompRet}\textbackslash item{\AutoCompRet}•{\AutoCompRet}\textbackslash end\{compactitem\}• \\
bdes            & \textbackslash bdes      & \textbackslash begin\{description\}{\AutoCompRet}\textbackslash item[{\AutoCompIns}]{\AutoCompRet}•{\AutoCompRet}\textbackslash end\{description\}• \\
bdoc            & \textbackslash bdoc      & \textbackslash begin\{document\}{\AutoCompRet}{\AutoCompRet}{\AutoCompIns}{\AutoCompRet}{\AutoCompRet}\textbackslash end\{document\} \\
benu            & \textbackslash benu      & \textbackslash begin\{enumerate\}{\AutoCompRet}\textbackslash item{\AutoCompRet}{\AutoCompIns}{\AutoCompRet}\textbackslash end\{enumerate\}• \\
benuo           & \textbackslash benuo     & \textbackslash begin\{enumerate\}[{\AutoCompIns}]{\AutoCompRet}\textbackslash item{\AutoCompRet}•{\AutoCompRet}\textbackslash end\{enumerate\}• \\
beqns           & \textbackslash beqns     & \textbackslash begin\{eqnarray*\}{\AutoCompRet}{\AutoCompIns}{\AutoCompRet}\textbackslash end\{eqnarray*\}• \\
beqn            & \textbackslash beqn      & \textbackslash begin\{eqnarray\}{\AutoCompRet}{\AutoCompIns}{\AutoCompRet}\textbackslash end\{eqnarray\}• \\
bequs           & \textbackslash bequs     & \textbackslash begin\{equation*\}{\AutoCompRet}{\AutoCompIns}{\AutoCompRet}\textbackslash end\{equation*\}• \\
bequ            & \textbackslash bequ      & \textbackslash begin\{equation\}{\AutoCompRet}{\AutoCompIns}{\AutoCompRet}\textbackslash end\{equation\}• \\
bfig            & \textbackslash bfig      & \textbackslash begin\{figure\}{\AutoCompRet}{\AutoCompIns}{\AutoCompRet}\textbackslash end\{figure\}• \\
bfigo           & \textbackslash bfigo     & \textbackslash begin\{figure\}[{\AutoCompIns}]{\AutoCompRet}•{\AutoCompRet}\textbackslash end\{figure\}• \\
bflaligs        & \textbackslash bflaligs  & \textbackslash begin\{flalign*\}{\AutoCompRet}{\AutoCompIns}{\AutoCompRet}\textbackslash end\{flalign*\}• \\
bflalig         & \textbackslash bflalig   & \textbackslash begin\{flalign\}{\AutoCompRet}{\AutoCompIns}{\AutoCompRet}\textbackslash end\{flalign\}• \\
bfll            & \textbackslash bfll      & \textbackslash begin\{flushleft\}{\AutoCompRet}{\AutoCompIns}{\AutoCompRet}\textbackslash end\{flushleft\}• \\
bflr            & \textbackslash bflr      & \textbackslash begin\{flushright\}{\AutoCompRet}{\AutoCompIns}{\AutoCompRet}\textbackslash end\{flushright\}• \\
bgaths          & \textbackslash bgaths    & \textbackslash begin\{gather*\}{\AutoCompRet}{\AutoCompIns}{\AutoCompRet}\textbackslash end\{gather*\}• \\
bgathed         & \textbackslash bgathed   & \textbackslash begin\{gathered\}{\AutoCompRet}{\AutoCompIns}{\AutoCompRet}\textbackslash end\{gathered\}• \\
bgathedo        & \textbackslash bgathedo  & \textbackslash begin\{gathered\}[{\AutoCompIns}]{\AutoCompRet}•{\AutoCompRet}\textbackslash end\{gathered\}• \\
bgath           & \textbackslash bgath     & \textbackslash begin\{gather\}{\AutoCompRet}{\AutoCompIns}{\AutoCompRet}\textbackslash end\{gather\}• \\
bite            & \textbackslash bite      & \textbackslash begin\{itemize\}{\AutoCompRet}\textbackslash item{\AutoCompRet}{\AutoCompIns}{\AutoCompRet}\textbackslash end\{itemize\}• \\
biteo           & \textbackslash biteo     & \textbackslash begin\{itemize\}[{\AutoCompIns}]{\AutoCompRet}\textbackslash item{\AutoCompRet}•{\AutoCompRet}\textbackslash end\{itemize\}• \\
blett           & \textbackslash blett     & \textbackslash begin\{letter\}\{{\AutoCompIns}\}{\AutoCompRet}•{\AutoCompRet}\textbackslash end\{letter\}• \\
blist           & \textbackslash blist     & \textbackslash begin\{list\}\{{\AutoCompIns}\}\{•\}{\AutoCompRet}\textbackslash item{\AutoCompRet}•{\AutoCompRet}\textbackslash end\{list\}• \\
bminpo          & \textbackslash bminpo    & \textbackslash begin\{minipage\}[{\AutoCompIns}]\{•\}{\AutoCompRet}•{\AutoCompRet}\textbackslash end\{minipage\}• \\
bminp           & \textbackslash bminp     & \textbackslash begin\{minipage\}\{{\AutoCompIns}\}{\AutoCompRet}•{\AutoCompRet}\textbackslash end\{minipage\}• \\
bmults          & \textbackslash bmults    & \textbackslash begin\{multline*\}{\AutoCompRet}{\AutoCompIns}{\AutoCompRet}\textbackslash end\{multline*\}• \\
bmult           & \textbackslash bmult     & \textbackslash begin\{multline\}{\AutoCompRet}{\AutoCompIns}{\AutoCompRet}\textbackslash end\{multline\}• \\
bpict           & \textbackslash bpict     & \textbackslash begin\{picture\}{\AutoCompRet}{\AutoCompIns}{\AutoCompRet}\textbackslash end\{picture\}• \\
bpmat           & \textbackslash bpmat     & \textbackslash begin\{pmatrix\}{\AutoCompRet}{\AutoCompIns}{\AutoCompRet}\textbackslash end\{pmatrix\}• \\
bquot           & \textbackslash bquot     & \textbackslash begin\{quotation\}{\AutoCompRet}{\AutoCompIns}{\AutoCompRet}\textbackslash end\{quotation\}• \\
bquo            & \textbackslash bquo      & \textbackslash begin\{quote\}{\AutoCompRet}{\AutoCompIns}{\AutoCompRet}\textbackslash end\{quote\}• \\
bsplit          & \textbackslash bsplit    & \textbackslash begin\{split\}{\AutoCompRet}{\AutoCompIns}{\AutoCompRet}\textbackslash end\{split\}• \\
bsubeq          & \textbackslash bsubeq    & \textbackslash begin\{subequations\}{\AutoCompRet}{\AutoCompIns}{\AutoCompRet}\textbackslash end\{subequations\}• \\
btabb           & \textbackslash btabb     & \textbackslash begin\{tabbing\}{\AutoCompRet}{\AutoCompIns}{\AutoCompRet}\textbackslash end\{tabbing\}• \\
btbls           & \textbackslash btbls     & \textbackslash begin\{table*\}{\AutoCompRet}{\AutoCompIns}{\AutoCompRet}\textbackslash end\{table*\}• \\
btabls          & \textbackslash btabls    & \textbackslash begin\{table*\}{\AutoCompRet}{\AutoCompIns}{\AutoCompRet}\textbackslash end\{table*\}• \\
btablso         & \textbackslash btablso   & \textbackslash begin\{table*\}[{\AutoCompIns}]{\AutoCompRet}•{\AutoCompRet}\textbackslash end\{table*\}• \\
btblso          & \textbackslash btblso    & \textbackslash begin\{table*\}[{\AutoCompIns}]{\AutoCompRet}•{\AutoCompRet}\textbackslash end\{table*\}• \\
btbl            & \textbackslash btbl      & \textbackslash begin\{table\}{\AutoCompRet}{\AutoCompIns}{\AutoCompRet}\textbackslash end\{table\}• \\
btabl           & \textbackslash btabl     & \textbackslash begin\{table\}{\AutoCompRet}{\AutoCompIns}{\AutoCompRet}\textbackslash end\{table\}• \\
btblo           & \textbackslash btblo     & \textbackslash begin\{table\}[{\AutoCompIns}]{\AutoCompRet}•{\AutoCompRet}\textbackslash end\{table\}• \\
btablo          & \textbackslash btablo    & \textbackslash begin\{table\}[{\AutoCompIns}]{\AutoCompRet}•{\AutoCompRet}\textbackslash end\{table\}• \\
btabs           & \textbackslash btabs     & \textbackslash begin\{tabular*\}\{{\AutoCompIns}\}\{•\}{\AutoCompRet}•{\AutoCompRet}\textbackslash end\{tabular*\}• \\
btabx           & \textbackslash btabx     & \textbackslash begin\{tabularx\}\{{\AutoCompIns}\}\{•\}{\AutoCompRet}•{\AutoCompRet}\textbackslash end\{tabularx\}• \\
btab            & \textbackslash btab      & \textbackslash begin\{tabular\}\{{\AutoCompIns}\}{\AutoCompRet}•{\AutoCompRet}\textbackslash end\{tabular\}• \\
bbib            & \textbackslash bbib      & \textbackslash begin\{thebibliography\}\{{\AutoCompIns}\}{\AutoCompRet}\textbackslash bibitem\{•\}{\AutoCompRet}•{\AutoCompRet}\textbackslash end\{thebibliography\}• \\
bindex          & \textbackslash bindex    & \textbackslash begin\{theindex\}{\AutoCompRet}{\AutoCompIns}{\AutoCompRet}\textbackslash end\{theindex\}• \\
btheo           & \textbackslash btheo     & \textbackslash begin\{theorem\}{\AutoCompRet}{\AutoCompIns}{\AutoCompRet}\textbackslash end\{theorem\}• \\
btitpg          & \textbackslash btitpg    & \textbackslash begin\{titlepage\}{\AutoCompRet}{\AutoCompIns}{\AutoCompRet}\textbackslash end\{titlepage\}• \\
btrivl          & \textbackslash btrivl    & \textbackslash begin\{trivlist\}{\AutoCompRet}{\AutoCompIns}{\AutoCompRet}\textbackslash end\{trivlist\}• \\
bvarw           & \textbackslash bvarw     & \textbackslash begin\{varwidth\}\{{\AutoCompIns}\}{\AutoCompRet}•{\AutoCompRet}\textbackslash end\{varwidth\}• \\
bverb           & \textbackslash bverb     & \textbackslash begin\{verbatim\}{\AutoCompRet}{\AutoCompIns}{\AutoCompRet}\textbackslash end\{verbatim\}• \\
bvers           & \textbackslash bvers     & \textbackslash begin\{verse\}{\AutoCompRet}{\AutoCompIns}{\AutoCompRet}\textbackslash end\{verse\}• \\
                &                          & \textbackslash bfseries \\
bfd             &                          & \textbackslash bfseries \\
bibitemo        &                          & \textbackslash bibitem[{\AutoCompIns}]\{•\}{\AutoCompRet}• \\
                &                          & \textbackslash bibitem[{\AutoCompIns}]\{•\}{\AutoCompRet}• \\
bibitem         &                          & \textbackslash bibitem\{{\AutoCompIns}\}{\AutoCompRet}• \\
                &                          & \textbackslash bibitem\{{\AutoCompIns}\}{\AutoCompRet}• \\
bibstyle        & \textbackslash bibstyle  & \textbackslash bibliographystyle\{{\AutoCompIns}\} \\
biblio          &                          & \textbackslash bibliography\{{\AutoCompIns}\} \\
                &                          & \textbackslash bibliography\{{\AutoCompIns}\} \\
                &                          & \textbackslash bottomrule{\AutoCompRet} \\
botr            &                          & \textbackslash bottomrule{\AutoCompRet} \\
                &                          & \textbackslash boxed\{{\AutoCompIns}\} \\
                &                          & \textbackslash caption\{{\AutoCompIns}\}{\AutoCompRet} \\
                &                          & \textbackslash cdots \\
center          &                          & \textbackslash centering \\
                &                          & \textbackslash centering \\
                &                          & \textbackslash chapter\{{\AutoCompIns}\} \\
chap            &                          & \textbackslash chapter\{{\AutoCompIns}\}{\AutoCompRet} \\
                &                          & \textbackslash citep\{{\AutoCompIns}\} \\
                &                          & \textbackslash citet\{{\AutoCompIns}\} \\
                &                          & \textbackslash cite\{{\AutoCompIns}\} \\
                &                          & \textbackslash cline\{{\AutoCompIns}\} \\
                &                          & \textbackslash cmidrule({\AutoCompIns})\{•\} \\
cmidr           &                          & \textbackslash cmidrule({\AutoCompIns})\{•\} \\
cmidro          &                          & \textbackslash cmidrule[{\AutoCompIns}](•)\{•\} \\
                &                          & \textbackslash cmidrule[{\AutoCompIns}](•)\{•\} \\
                &                          & \textbackslash date\{{\AutoCompIns}\}{\AutoCompRet} \\
                &                          & \textbackslash ddddot\{{\AutoCompIns}\} \\
                &                          & \textbackslash dddot\{{\AutoCompIns}\} \\
                &                          & \textbackslash ddots \\
                &                          & \textbackslash ddot\{{\AutoCompIns}\} \\
                &                          & \textbackslash documentclass[{\AutoCompIns}]\{•\}{\AutoCompRet} \\
                &                          & \textbackslash documentclass\{{\AutoCompIns}\}{\AutoCompRet} \\
                &                          & \textbackslash dots \\
                &                          & \textbackslash dotsb \\
                &                          & \textbackslash dotsc \\
                &                          & \textbackslash dotsi \\
                &                          & \textbackslash dotsm \\
                &                          & \textbackslash dotso \\
emd             &                          & \textbackslash em \\
em              &                          & \textbackslash emph\{{\AutoCompIns}\} \\
                &                          & \textbackslash emph\{{\AutoCompIns}\} \\
                &                          & \textbackslash end\{{\AutoCompIns}\}{\AutoCompRet} \\
                &                          & \textbackslash eqref\{{\AutoCompIns}\} \\
                &                          & \textbackslash fboxrule\{{\AutoCompIns}\} \\
                &                          & \textbackslash fboxsep\{{\AutoCompIns}\} \\
fbox            &                          & \textbackslash fbox\{{\AutoCompIns}\} \\
                &                          & \textbackslash fbox\{{\AutoCompIns}\} \\
                &                          & \textbackslash footnotesize \\
foot            &                          & \textbackslash footnote\{{\AutoCompIns}\} \\
                &                          & \textbackslash footnote\{{\AutoCompIns}\} \\
frac            &                          & \textbackslash frac\{{\AutoCompIns}\}\{•\} \\
                &                          & \textbackslash frac\{{\AutoCompIns}\}\{•\} \\
fboxoo          & \textbackslash fboxoo    & \textbackslash framebox[{\AutoCompIns}][•]\{•\} \\
                &                          & \textbackslash framebox[{\AutoCompIns}][•]\{•\} \\
                &                          & \textbackslash framebox[{\AutoCompIns}]\{•\} \\
fboxo           & \textbackslash fboxo     & \textbackslash framebox[{\AutoCompIns}]\{•\} \\
geometry        &                          & \textbackslash geometry\{•\} \\
                &                          & \textbackslash geometry\{•\} \\
                &                          & \textbackslash headwidth \\
hw              &                          & \textbackslash headwidth \\
                &                          & \textbackslash hline{\AutoCompRet} \\
href            &                          & \textbackslash href\{{\AutoCompIns}\}\{•\} \\
                &                          & \textbackslash href\{{\AutoCompIns}\}\{•\} \\
                &                          & \textbackslash hspace*\{{\AutoCompIns}\} \\
                &                          & \textbackslash hspace\{{\AutoCompIns}\} \\
incgo           &                          & \textbackslash includegraphics[{\AutoCompIns}]\{•\}{\AutoCompRet} \\
                &                          & \textbackslash includegraphics[{\AutoCompIns}]\{•\}{\AutoCompRet} \\
                &                          & \textbackslash includegraphics\{{\AutoCompIns}\}{\AutoCompRet} \\
incg            &                          & \textbackslash includegraphics\{{\AutoCompIns}\}{\AutoCompRet} \\
                &                          & \textbackslash include\{{\AutoCompIns}\}{\AutoCompRet} \\
                &                          & \textbackslash intertext\{{\AutoCompIns}\} \\
                &                          & \textbackslash item{\AutoCompRet}{\AutoCompIns} \\
ito             &                          & \textbackslash item[{\AutoCompIns}]{\AutoCompRet}• \\
                &                          & \textbackslash item[{\AutoCompIns}]{\AutoCompRet}• \\
itd             &                          & \textbackslash itshape \\
                &                          & \textbackslash itshape \\
lbl             & \textbackslash lbl       & \textbackslash label\{{\AutoCompIns}\} \\
                &                          & \textbackslash label\{{\AutoCompIns}\} \\
                &                          & \textbackslash Large \\
                &                          & \textbackslash large \\
                &                          & \textbackslash LaTeX \\
latex           & \textbackslash latex     & \textbackslash LaTeX \\
                &                          & \textbackslash LaTeX\textbackslash  \\
latexs          & \textbackslash latexs    & \textbackslash LaTeX\textbackslash  \\
                &                          & \textbackslash LaTeXe \\
latexe          & \textbackslash latexe    & \textbackslash LaTeXe \\
latexes         & \textbackslash latexes   & \textbackslash LaTeXe\textbackslash  \\
                &                          & \textbackslash LaTeXe\textbackslash  \\
                &                          & \textbackslash ldots \\
                &                          & \textbackslash listoffigures{\AutoCompRet} \\
listf           & \textbackslash listf     & \textbackslash listoffigures{\AutoCompRet} \\
                &                          & \textbackslash listoftables{\AutoCompRet} \\
listt           & \textbackslash listt     & \textbackslash listoftables{\AutoCompRet} \\
mboxoo          & \textbackslash mboxoo    & \textbackslash makebox[{\AutoCompIns}][•]\{•\} \\
                &                          & \textbackslash makebox[{\AutoCompIns}][•]\{•\} \\
                &                          & \textbackslash makebox[{\AutoCompIns}]\{•\} \\
mboxo           & \textbackslash mboxo     & \textbackslash makebox[{\AutoCompIns}]\{•\} \\
mpar            & \textbackslash mpar      & \textbackslash marginpar\{{\AutoCompIns}\} \\
                &                          & \textbackslash marginpar\{{\AutoCompIns}\} \\
                &                          & \textbackslash mathbf\{{\AutoCompIns}\} \\
mbf             & \textbackslash mbf       & \textbackslash mathbf\{{\AutoCompIns}\} \\
mcal            & \textbackslash mcal      & \textbackslash mathcal\{{\AutoCompIns}\} \\
                &                          & \textbackslash mathcal\{{\AutoCompIns}\} \\
mit             & \textbackslash mit       & \textbackslash mathit\{{\AutoCompIns}\} \\
                &                          & \textbackslash mathit\{{\AutoCompIns}\} \\
mnorm           & \textbackslash mnorm     & \textbackslash mathnormal\{{\AutoCompIns}\} \\
                &                          & \textbackslash mathnormal\{{\AutoCompIns}\} \\
                &                          & \textbackslash mathrm\{{\AutoCompIns}\} \\
mrm             & \textbackslash mrm       & \textbackslash mathrm\{{\AutoCompIns}\} \\
msf             & \textbackslash msf       & \textbackslash mathsf\{{\AutoCompIns}\} \\
                &                          & \textbackslash mathsf\{{\AutoCompIns}\} \\
                &                          & \textbackslash mathtt\{{\AutoCompIns}\} \\
mtt             & \textbackslash mtt       & \textbackslash mathtt\{{\AutoCompIns}\} \\
mbox            &                          & \textbackslash mbox\{{\AutoCompIns}\} \\
                &                          & \textbackslash mbox\{{\AutoCompIns}\} \\
                &                          & \textbackslash mdseries \\
mdd             &                          & \textbackslash mdseries \\
                &                          & \textbackslash midrule{\AutoCompRet} \\
midr            &                          & \textbackslash midrule{\AutoCompRet} \\
multc           & \textbackslash multc     & \textbackslash multicolumn\{{\AutoCompIns}\}\{•\}\{•\} \\
                &                          & \textbackslash multicolumn\{{\AutoCompIns}\}\{•\}\{•\} \\
multic          &                          & \textbackslash multicolumn\{{\AutoCompIns}\}\{•\}\{•\} \\
nct             &                          & \textbackslash newcolumntype\{{\AutoCompIns}\}\{•\} \\
newct           &                          & \textbackslash newcolumntype\{{\AutoCompIns}\}\{•\} \\
                &                          & \textbackslash newcolumntype\{{\AutoCompIns}\}\{•\} \\
                &                          & \textbackslash newcommand\{{\AutoCompIns}\}[•][•]\{•\}{\AutoCompRet} \\
ncmoo           &                          & \textbackslash newcommand\{{\AutoCompIns}\}[•][•]\{•\}{\AutoCompRet} \\
newcoo          &                          & \textbackslash newcommand\{{\AutoCompIns}\}[•][•]\{•\}{\AutoCompRet} \\
newco           &                          & \textbackslash newcommand\{{\AutoCompIns}\}[•]\{•\}{\AutoCompRet} \\
ncmo            &                          & \textbackslash newcommand\{{\AutoCompIns}\}[•]\{•\}{\AutoCompRet} \\
                &                          & \textbackslash newcommand\{{\AutoCompIns}\}[•]\{•\}{\AutoCompRet} \\
                &                          & \textbackslash newcommand\{{\AutoCompIns}\}\{•\}{\AutoCompRet} \\
ncm             &                          & \textbackslash newcommand\{{\AutoCompIns}\}\{•\}{\AutoCompRet} \\
newc            &                          & \textbackslash newcommand\{{\AutoCompIns}\}\{•\}{\AutoCompRet} \\
nenvoo          &                          & \textbackslash newenvironment\{{\AutoCompIns}\}[•][•]\{•\}\{•\}{\AutoCompRet} \\
                &                          & \textbackslash newenvironment\{{\AutoCompIns}\}[•][•]\{•\}\{•\}{\AutoCompRet} \\
neweoo          &                          & \textbackslash newenvironment\{{\AutoCompIns}\}[•][•]\{•\}\{•\}{\AutoCompRet} \\
nenvo           &                          & \textbackslash newenvironment\{{\AutoCompIns}\}[•]\{•\}\{•\}{\AutoCompRet} \\
neweo           &                          & \textbackslash newenvironment\{{\AutoCompIns}\}[•]\{•\}\{•\}{\AutoCompRet} \\
nenv            &                          & \textbackslash newenvironment\{{\AutoCompIns}\}\{•\}\{•\}{\AutoCompRet} \\
newe            &                          & \textbackslash newenvironment\{{\AutoCompIns}\}\{•\}\{•\}{\AutoCompRet} \\
                &                          & \textbackslash newenvironment\{{\AutoCompIns}\}\{•\}\{•\}{\AutoCompRet} \\
nlen            &                          & \textbackslash newlength\{{\AutoCompIns}\}{\AutoCompRet} \\
                &                          & \textbackslash newlength\{{\AutoCompIns}\}{\AutoCompRet} \\
newlen          &                          & \textbackslash newlength\{{\AutoCompIns}\}{\AutoCompRet} \\
newlin          &                          & \textbackslash newline{\AutoCompRet} \\
nline           &                          & \textbackslash newline{\AutoCompRet} \\
                &                          & \textbackslash newline{\AutoCompRet} \\
npg             & \textbackslash npg       & \textbackslash newpage{\AutoCompRet} \\
newpg           &                          & \textbackslash newpage{\AutoCompRet} \\
                &                          & \textbackslash newpage{\AutoCompRet} \\
                &                          & \textbackslash newtheorem\{{\AutoCompIns}\}[•]\{•\}{\AutoCompRet} \\
                &                          & \textbackslash newtheorem\{{\AutoCompIns}\}\{•\}{\AutoCompRet} \\
                &                          & \textbackslash newtheorem\{{\AutoCompIns}\}\{•\}[•]{\AutoCompRet} \\
                &                          & \textbackslash nocite\{{\AutoCompIns}\} \\
                &                          & \textbackslash normalsize \\
                &                          & \textbackslash pagebreak{\AutoCompRet} \\
pgref           &                          & \textbackslash pageref\{{\AutoCompIns}\} \\
                &                          & \textbackslash pageref\{{\AutoCompIns}\} \\
pgs             &                          & \textbackslash pagestyle\{{\AutoCompIns}\}{\AutoCompRet} \\
                &                          & \textbackslash pagestyle\{{\AutoCompIns}\}{\AutoCompRet} \\
pars            &                          & \textbackslash paragraph*\{{\AutoCompIns}\}{\AutoCompRet} \\
                &                          & \textbackslash paragraph*\{{\AutoCompIns}\}{\AutoCompRet} \\
                &                          & \textbackslash paragraph[{\AutoCompIns}]\{•\}{\AutoCompRet} \\
paro            &                          & \textbackslash paragraph[{\AutoCompIns}]\{•\}{\AutoCompRet} \\
                &                          & \textbackslash paragraph\{{\AutoCompIns}\}{\AutoCompRet} \\
par             &                          & \textbackslash paragraph\{{\AutoCompIns}\}{\AutoCompRet} \\
pboxo           & \textbackslash pboxo     & \textbackslash parbox[{\AutoCompIns}]\{•\}\{•\} \\
                &                          & \textbackslash parbox[{\AutoCompIns}]\{•\}\{•\} \\
parboxo         &                          & \textbackslash parbox[{\AutoCompIns}]\{•\}\{•\} \\
parbox          &                          & \textbackslash parbox\{{\AutoCompIns}\}\{•\} \\
                &                          & \textbackslash parbox\{{\AutoCompIns}\}\{•\} \\
                & \textbackslash pbox      & \textbackslash parbox\{{\AutoCompIns}\}\{•\} \\
                &                          & \textbackslash pbox\{{\AutoCompIns}\}\{•\} \\
pbox            &                          & \textbackslash pbox\{\#INS\}\{•\} \\
rboxoo          & \textbackslash rboxoo    & \textbackslash raisebox\{{\AutoCompIns}\}[•][•]\{•\} \\
                &                          & \textbackslash raisebox\{{\AutoCompIns}\}[•][•]\{•\} \\
rboxo           & \textbackslash rboxo     & \textbackslash raisebox\{{\AutoCompIns}\}[•]\{•\} \\
                &                          & \textbackslash raisebox\{{\AutoCompIns}\}[•]\{•\} \\
                &                          & \textbackslash raisebox\{{\AutoCompIns}\}\{•\} \\
rbox            & \textbackslash rbox      & \textbackslash raisebox\{{\AutoCompIns}\}\{•\} \\
ref             &                          & \textbackslash ref\{{\AutoCompIns}\} \\
                &                          & \textbackslash ref\{{\AutoCompIns}\} \\
rnewcoo         &                          & \textbackslash renewcommand\{{\AutoCompIns}\}[•][•]\{•\}{\AutoCompRet} \\
rncmoo          &                          & \textbackslash renewcommand\{{\AutoCompIns}\}[•][•]\{•\}{\AutoCompRet} \\
                &                          & \textbackslash renewcommand\{{\AutoCompIns}\}[•][•]\{•\}{\AutoCompRet} \\
rnewco          &                          & \textbackslash renewcommand\{{\AutoCompIns}\}[•]\{•\}{\AutoCompRet} \\
rncmo           &                          & \textbackslash renewcommand\{{\AutoCompIns}\}[•]\{•\}{\AutoCompRet} \\
                &                          & \textbackslash renewcommand\{{\AutoCompIns}\}[•]\{•\}{\AutoCompRet} \\
rncm            &                          & \textbackslash renewcommand\{{\AutoCompIns}\}\{•\}{\AutoCompRet} \\
rnewc           &                          & \textbackslash renewcommand\{{\AutoCompIns}\}\{•\}{\AutoCompRet} \\
                &                          & \textbackslash renewcommand\{{\AutoCompIns}\}\{•\}{\AutoCompRet} \\
                &                          & \textbackslash rmfamily \\
rmc             &                          & \textbackslash rmfamily \\
                &                          & \textbackslash rule[{\AutoCompIns}]\{•\}\{•\} \\
                &                          & \textbackslash rule\{{\AutoCompIns}\}\{•\} \\
scd             &                          & \textbackslash scshape \\
                &                          & \textbackslash scshape \\
secs            &                          & \textbackslash section*\{{\AutoCompIns}\}{\AutoCompRet} \\
                &                          & \textbackslash section*\{{\AutoCompIns}\}{\AutoCompRet} \\
seco            &                          & \textbackslash section[{\AutoCompIns}]\{•\}{\AutoCompRet} \\
                &                          & \textbackslash section[{\AutoCompIns}]\{•\}{\AutoCompRet} \\
                &                          & \textbackslash section\{{\AutoCompIns}\}{\AutoCompRet} \\
sec             &                          & \textbackslash section\{{\AutoCompIns}\}{\AutoCompRet} \\
                &                          & \textbackslash setlength\{{\AutoCompIns}\}\{•\} \\
hw2tw           &                          & \textbackslash setlength\{\textbackslash headwidth\}\{\textbackslash textwidth\}{\AutoCompRet} \\
                &                          & \textbackslash sffamily \\
sfd             &                          & \textbackslash sffamily \\
                &                          & \textbackslash slshape \\
sld             &                          & \textbackslash slshape \\
sqrto           & \textbackslash sqrto     & \textbackslash sqrt[{\AutoCompIns}]\{•\} \\
sqrt            & \textbackslash sqrt      & \textbackslash sqrt\{{\AutoCompIns}\} \\
stcount         &                          & \textbackslash stepcounter\{{\AutoCompIns}\}{\AutoCompRet} \\
spars           & \textbackslash spars     & \textbackslash subparagraph*\{{\AutoCompIns}\} \\
                &                          & \textbackslash subparagraph*\{{\AutoCompIns}\} \\
                &                          & \textbackslash subparagraph[{\AutoCompIns}]\{•\} \\
sparo           & \textbackslash sparo     & \textbackslash subparagraph[{\AutoCompIns}]\{•\} \\
                &                          & \textbackslash subparagraph\{{\AutoCompIns}\} \\
spar            & \textbackslash spar      & \textbackslash subparagraph\{{\AutoCompIns}\} \\
ssecs           & \textbackslash ssecs     & \textbackslash subsection*\{{\AutoCompIns}\}{\AutoCompRet} \\
                &                          & \textbackslash subsection*\{{\AutoCompIns}\}{\AutoCompRet} \\
sseco           & \textbackslash sseco     & \textbackslash subsection[{\AutoCompIns}]\{•\}{\AutoCompRet} \\
                &                          & \textbackslash subsection[{\AutoCompIns}]\{•\}{\AutoCompRet} \\
ssec            & \textbackslash ssec      & \textbackslash subsection\{{\AutoCompIns}\}{\AutoCompRet} \\
                &                          & \textbackslash subsection\{{\AutoCompIns}\}{\AutoCompRet} \\
                &                          & \textbackslash subsubsection*\{{\AutoCompIns}\}{\AutoCompRet} \\
sssecs          & \textbackslash sssecs    & \textbackslash subsubsection*\{{\AutoCompIns}\}{\AutoCompRet} \\
                &                          & \textbackslash subsubsection[{\AutoCompIns}][•]{\AutoCompRet} \\
                & \textbackslash ssseco    & \textbackslash subsubsection[{\AutoCompIns}][•]{\AutoCompRet} \\
ssseco          &                          & \textbackslash subsubsection[{\AutoCompIns}]\{•\}{\AutoCompRet} \\
sssec           & \textbackslash sssec     & \textbackslash subsubsection\{{\AutoCompIns}\}{\AutoCompRet} \\
                &                          & \textbackslash subsubsection\{{\AutoCompIns}\}{\AutoCompRet} \\
toc             & \textbackslash toc       & \textbackslash tableofcontents{\AutoCompRet} \\
tableofcontents &                          & \textbackslash tableofcontents{\AutoCompRet} \\
                &                          & \textbackslash tableofcontents{\AutoCompRet} \\
tilde           & \textbackslash tilde     & \textbackslash textasciitilde \\
bf              & \textbackslash bf        & \textbackslash textbf\{{\AutoCompIns}\} \\
                &                          & \textbackslash textbf\{{\AutoCompIns}\} \\
{-}{-}{-}       &                          & \textbackslash textemdash\textbackslash  \\
{-}{-}          &                          & \textbackslash textendash\textbackslash  \\
it              & \textbackslash it        & \textbackslash textit\{{\AutoCompIns}\} \\
                &                          & \textbackslash textit\{{\AutoCompIns}\} \\
                & \textbackslash rm        & \textbackslash textrm\{{\AutoCompIns}\} \\
sc              & \textbackslash sc        & \textbackslash textsc\{{\AutoCompIns}\} \\
                &                          & \textbackslash textsc\{{\AutoCompIns}\} \\
sf              & \textbackslash sf        & \textbackslash textsf\{{\AutoCompIns}\} \\
                &                          & \textbackslash textsf\{{\AutoCompIns}\} \\
                &                          & \textbackslash textsl\{{\AutoCompIns}\} \\
sl              & \textbackslash sl        & \textbackslash textsl\{{\AutoCompIns}\} \\
tt              & \textbackslash tt        & \textbackslash texttt\{{\AutoCompIns}\} \\
                &                          & \textbackslash texttt\{{\AutoCompIns}\} \\
                &                          & \textbackslash textup\{{\AutoCompIns}\} \\
up              & \textbackslash up        & \textbackslash textup\{{\AutoCompIns}\} \\
tw              & \textbackslash tw        & \textbackslash textwidth \\
                &                          & \textbackslash textwidth \\
                &                          & \textbackslash text\{{\AutoCompIns}\} \\
                &                          & \textbackslash thanks\{{\AutoCompIns}\}{\AutoCompRet} \\
                &                          & \textbackslash title\{{\AutoCompIns}\}{\AutoCompRet} \\
topr            &                          & \textbackslash toprule{\AutoCompRet} \\
                &                          & \textbackslash toprule{\AutoCompRet} \\
ttd             &                          & \textbackslash ttfamily \\
                &                          & \textbackslash ttfamily \\
upd             &                          & \textbackslash upshape \\
                &                          & \textbackslash upshape \\
url             &                          & \textbackslash url\{{\AutoCompIns}\} \\
                &                          & \textbackslash url\{{\AutoCompIns}\} \\
                &                          & \textbackslash usepackage[{\AutoCompIns}]\{•\}{\AutoCompRet} \\
usepo           &                          & \textbackslash usepackage[{\AutoCompIns}]\{•\}{\AutoCompRet} \\
usep            &                          & \textbackslash usepackage\{{\AutoCompIns}\}{\AutoCompRet} \\
                &                          & \textbackslash usepackage\{{\AutoCompIns}\}{\AutoCompRet} \\
                &                          & \textbackslash vdots \\
                &                          & \textbackslash vspace*\{{\AutoCompIns}\}{\AutoCompRet} \\
                &                          & \textbackslash vspace\{{\AutoCompIns}\}{\AutoCompRet} \\
                &                          & \{abstract\}{\AutoCompRet}{\AutoCompIns}{\AutoCompRet}\textbackslash end\{abstract\}• \\
                &                          & \{align*\}{\AutoCompRet}{\AutoCompIns}{\AutoCompRet}\textbackslash end\{align*\}• \\
                &                          & \{alignat*\}\{{\AutoCompIns}\}{\AutoCompRet}•{\AutoCompRet}\textbackslash end\{alignat*\}• \\
                &                          & \{alignat\}\{{\AutoCompIns}\}{\AutoCompRet}•{\AutoCompRet}\textbackslash end\{alignat\}• \\
                &                          & \{alignedat\}\{{\AutoCompIns}\}{\AutoCompRet}•{\AutoCompRet}\textbackslash end\{alignedat\}• \\
                &                          & \{aligned\}{\AutoCompRet}{\AutoCompIns}{\AutoCompRet}\textbackslash end\{aligned\}• \\
                &                          & \{aligned\}[{\AutoCompIns}]{\AutoCompRet}•{\AutoCompRet}\textbackslash end\{aligned\}• \\
                &                          & \{align\}{\AutoCompRet}{\AutoCompIns}{\AutoCompRet}\textbackslash end\{align\}• \\
                &                          & \{appendix\}{\AutoCompRet}{\AutoCompIns}{\AutoCompRet}\textbackslash end\{appendix\}• \\
                &                          & \{array\}{\AutoCompRet}{\AutoCompIns}{\AutoCompRet}\textbackslash end\{array\}• \\
                &                          & \{bmatrix\}{\AutoCompRet}{\AutoCompIns}{\AutoCompRet}\textbackslash end\{bmatrix\}• \\
                &                          & \{cases\}{\AutoCompRet}{\AutoCompIns}{\AutoCompRet}\textbackslash end\{cases\}• \\
                &                          & \{center\}{\AutoCompRet}{\AutoCompIns}{\AutoCompRet}\textbackslash end\{center\}• \\
                &                          & \{compactenum\}{\AutoCompRet}\textbackslash item{\AutoCompRet}{\AutoCompIns}{\AutoCompRet}\textbackslash end\{compactenum\}• \\
                &                          & \{compactenum\}[{\AutoCompIns}]{\AutoCompRet}\textbackslash item{\AutoCompRet}•{\AutoCompRet}\textbackslash end\{compactenum\}• \\
                &                          & \{compactitem\}{\AutoCompRet}\textbackslash item{\AutoCompRet}{\AutoCompIns}{\AutoCompRet}\textbackslash end\{compactitem\}• \\
                &                          & \{compactitem\}[{\AutoCompIns}]{\AutoCompRet}\textbackslash item{\AutoCompRet}•{\AutoCompRet}\textbackslash end\{compactitem\}• \\
                &                          & \{description\}{\AutoCompRet}\textbackslash item[{\AutoCompIns}]{\AutoCompRet}•{\AutoCompRet}\textbackslash end\{description\}• \\
                &                          & \{document\}{\AutoCompRet}{\AutoCompRet}{\AutoCompIns}{\AutoCompRet}{\AutoCompRet}\textbackslash end\{document\} \\
                &                          & \{enumerate\}{\AutoCompRet}\textbackslash item{\AutoCompRet}{\AutoCompIns}{\AutoCompRet}\textbackslash end\{enumerate\}• \\
                &                          & \{enumerate\}[{\AutoCompIns}]{\AutoCompRet}\textbackslash item{\AutoCompRet}•{\AutoCompRet}\textbackslash end\{enumerate\}• \\
                &                          & \{eqnarray*\}{\AutoCompRet}{\AutoCompIns}{\AutoCompRet}\textbackslash end\{eqnarray*\}• \\
                &                          & \{eqnarray\}{\AutoCompRet}{\AutoCompIns}{\AutoCompRet}\textbackslash end\{eqnarray\}• \\
                &                          & \{equation\}{\AutoCompRet}{\AutoCompIns}{\AutoCompRet}\textbackslash end\{equation\}• \\
                &                          & \{figure\}{\AutoCompRet}{\AutoCompIns}{\AutoCompRet}\textbackslash end\{figure\}• \\
                &                          & \{figure\}[{\AutoCompIns}]{\AutoCompRet}•{\AutoCompRet}\textbackslash end\{figure\}• \\
                &                          & \{flalign*\}{\AutoCompRet}{\AutoCompIns}{\AutoCompRet}\textbackslash end\{flalign*\}• \\
                &                          & \{flalign\}{\AutoCompRet}{\AutoCompIns}{\AutoCompRet}\textbackslash end\{flalign\}• \\
                &                          & \{flushleft\}{\AutoCompRet}{\AutoCompIns}{\AutoCompRet}\textbackslash end\{flushleft\}• \\
                &                          & \{flushright\}{\AutoCompRet}{\AutoCompIns}{\AutoCompRet}\textbackslash end\{flushright\}• \\
                &                          & \{gather*\}{\AutoCompRet}{\AutoCompIns}{\AutoCompRet}\textbackslash end\{gather*\}• \\
                &                          & \{gathered\}{\AutoCompRet}{\AutoCompIns}{\AutoCompRet}\textbackslash end\{gathered\}• \\
                &                          & \{gathered\}[{\AutoCompIns}]{\AutoCompRet}•{\AutoCompRet}\textbackslash end\{gathered\}• \\
                &                          & \{gather\}{\AutoCompRet}{\AutoCompIns}{\AutoCompRet}\textbackslash end\{gather\}• \\
                &                          & \{itemize\}{\AutoCompRet}\textbackslash item{\AutoCompRet}{\AutoCompIns}{\AutoCompRet}\textbackslash end\{itemize\}• \\
                &                          & \{itemize\}[{\AutoCompIns}]{\AutoCompRet}\textbackslash item{\AutoCompRet}•{\AutoCompRet}\textbackslash end\{itemize\}• \\
                &                          & \{letter\}\{{\AutoCompIns}\}{\AutoCompRet}•{\AutoCompRet}\textbackslash end\{letter\}• \\
                &                          & \{list\}\{{\AutoCompIns}\}\{•\}{\AutoCompRet}\textbackslash item{\AutoCompRet}•{\AutoCompRet}\textbackslash end\{list\}• \\
                &                          & \{minipage\}[{\AutoCompIns}]\{•\}{\AutoCompRet}•{\AutoCompRet}\textbackslash end\{minipage\}• \\
                &                          & \{minipage\}\{{\AutoCompIns}\}{\AutoCompRet}•{\AutoCompRet}\textbackslash end\{minipage\}• \\
                &                          & \{multline*\}{\AutoCompRet}{\AutoCompIns}{\AutoCompRet}\textbackslash end\{multline*\}• \\
                &                          & \{multline\}{\AutoCompRet}{\AutoCompIns}{\AutoCompRet}\textbackslash end\{multline\}• \\
                &                          & \{picture\}{\AutoCompRet}{\AutoCompIns}{\AutoCompRet}\textbackslash end\{picture\}• \\
                &                          & \{pmatrix\}{\AutoCompRet}{\AutoCompIns}{\AutoCompRet}\textbackslash end\{pmatrix\}• \\
                &                          & \{quotation\}{\AutoCompRet}{\AutoCompIns}{\AutoCompRet}\textbackslash end\{quotation\}• \\
                &                          & \{quote\}{\AutoCompRet}{\AutoCompIns}{\AutoCompRet}\textbackslash end\{quote\}• \\
                &                          & \{split\}{\AutoCompRet}{\AutoCompIns}{\AutoCompRet}\textbackslash end\{split\}• \\
                &                          & \{subequations\}{\AutoCompRet}{\AutoCompIns}{\AutoCompRet}\textbackslash end\{subequations\}• \\
                &                          & \{tabbing\}{\AutoCompRet}{\AutoCompIns}{\AutoCompRet}\textbackslash end\{tabbing\}• \\
                &                          & \{table*\}{\AutoCompRet}{\AutoCompIns}{\AutoCompRet}\textbackslash end\{table*\}• \\
                &                          & \{table*\}[{\AutoCompIns}]{\AutoCompRet}•{\AutoCompRet}\textbackslash end\{table*\}• \\
                &                          & \{table\}{\AutoCompRet}{\AutoCompIns}{\AutoCompRet}\textbackslash end\{table\}• \\
                &                          & \{table\}[{\AutoCompIns}]{\AutoCompRet}•{\AutoCompRet}\textbackslash end\{table\}• \\
                &                          & \{tabular*\}\{{\AutoCompIns}\}\{•\}{\AutoCompRet}•{\AutoCompRet}\textbackslash end\{tabular*\}• \\
                &                          & \{tabularx\}\{{\AutoCompIns}\}\{•\}{\AutoCompRet}•{\AutoCompRet}\textbackslash end\{tabularx\}• \\
                &                          & \{tabular\}\{{\AutoCompIns}\}{\AutoCompRet}•{\AutoCompRet}\textbackslash end\{tabular\}• \\
                &                          & \{thebibliography\}{\AutoCompRet}{\AutoCompIns}{\AutoCompRet}\textbackslash end\{thebibliography\}• \\
                &                          & \{theindex\}{\AutoCompRet}{\AutoCompIns}{\AutoCompRet}\textbackslash end\{theindex\}• \\
                &                          & \{theorem\}{\AutoCompRet}{\AutoCompIns}{\AutoCompRet}\textbackslash end\{theorem\}• \\
                &                          & \{titlepage\}{\AutoCompRet}{\AutoCompIns}{\AutoCompRet}\textbackslash end\{titlepage\}• \\
                &                          & \{trivlist\}{\AutoCompRet}{\AutoCompIns}{\AutoCompRet}\textbackslash end\{trivlist\}• \\
                &                          & \{varwidth\}\{{\AutoCompIns}\}{\AutoCompRet}•{\AutoCompRet}\textbackslash end\{varwidth\}• \\
                &                          & \{verbatim\}{\AutoCompRet}{\AutoCompIns}{\AutoCompRet}\textbackslash end\{verbatim\}• \\
                &                          & \{verse\}{\AutoCompRet}{\AutoCompIns}{\AutoCompRet}\textbackslash end\{verse\}• \\
\bottomrule
\end{longtable}


There are also environment codes (above) without \verb|\begin{| (which is itself a keyword); this allows to finish the environment name alone by \keysequence{Tab} if one started to input it manually.

