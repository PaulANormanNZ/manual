% !TEX encoding   = UTF8
% !TEX root       = manual.tex
% !TEX spellcheck = en_UK

\chapter{Compiling {\Tw}}\index{compiling {\Tw}}
\label{sec.compiling}

A complete guide how to compile {\Tw} is far beyond the scope of this manual. However, most users should find precompiled versions suitable for their system come either with their {\TeX} distribution or their operating system. If this is not the case, several precompiled versions can also be downloaded from \url{http://www.tug.org/texworks/}.

Compiling {\Tw} yourself is only necessary if your system is not (yet) supported, if you want to always have the latest features (and bugs), or generally want to help in improving {\Tw} further. To this end, there are some documents giving detailed instructions to compile {\Tw} on different machines.

\begin{OSLinux}
\noindent\url{https://github.com/TeXworks/texworks/wiki/Building} \\
\end{OSLinux}

\begin{OSMac}
\noindent\raggedright\url{https://github.com/TeXworks/texworks/wiki/Building-on-Mac-OS-X-(Homebrew)} \\
\end{OSMac}

\begin{OSWindows}
\noindent\raggedright\url{https://github.com/TeXworks/texworks/wiki/Building-on-Windows-(MinGW)} \\
\end{OSWindows}
