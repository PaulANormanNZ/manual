% !TEX encoding   = UTF8
% !TEX root       = manuel.tex
% !TEX spellcheck = fr_FR

\chapter{Au-del{\`a} de ce manuel}

Dans ce manuel, les auteurs ont essayé de donner un aperçu de \Tw{} et une introduction concise pour vous permettre de démarrer. Cependant, \Tw{} évolue et est amélioré continuellement, dès lors l'information présentée ici ne sera jamais complète.

Des fichiers additionnels et fréquemment mis-à-jour sont postés sur le wiki hébergé par Google Code à \url{http://code.google.com/p/texworks/w/list}. Les pages suivantes sont particulièrement à retenir:
\begin{description}
\item[SpellingDictionaries] décrit comment obtenir et installer des dictionnaires pour la correction orthographique sur divers systèmes. \url{http://code.google.com/p/texworks/wiki/SpellingDictionaries}
\item[TipsAndTricks] fournit une compilation de choses intéressantes à savoir en un coup d'œil, tel que le \verb|% !TEX root| construct. \url{http://code.google.com/p/texworks/wiki/TipsAndTricks}
\item[AdvancedTypesettingTools] liste les configurations de plusieurs outils de composition qui ne sont pas inclus par défaut dans \Tw, tels que latexmk ou la chaîne d'exécution dvips. \url{http://code.google.com/p/texworks/wiki/AdvancedTypesettingTools}
\end{description}

Si vous rencontrez un problème avec \Tw, il est conseillé de parcourir les archives de la liste de discussion accessible via \url{http://tug.org/pipermail/texworks/}. Si vous utilisez \Tw{} régulièrement ou êtes intéressés à apprendre au sujet des problèmes et solutions pour son utilisation pour tout autre raison, vous pourriez aussi envisager de vous inscrire à la liste \url{http://tug.org/mailman/listinfo/texworks} pour rester à jour. Pour des messages occasionnels à la liste, vous pouvez aussi utiliser l'option de menu \menu{Aide}\submenu\menu{Envoyer un courriel à la liste de discussion}. Veuillez, s'il-vous-plaît, à remplacer le sujet par défaut par quelque chose qui décrit votre problème et à inclure toute information qui peut aider à le résoudre. De cette façon, vous avez plus de chance de recevoir des réponses utiles.

Si vous trouvez un bogue dans \Tw{} ou si vous voulez proposer une nouvelle caractéristique pour une version ultérieure, jetez un coup d'œil à la liste des problèmes à Google Code (\url{http://code.google.com/p/texworks/issues/list}). Avant de poster un nouvel item,, s'il-vous-plaît, veuillez cependant vérifier si un rapport ou une demande identique n'existe pas déjà dans la liste et si la liste est la meilleure place pour cela. Si vous hésitez, posez la question à la liste de discussion auparavant.

Très bon \TeX{}age!
